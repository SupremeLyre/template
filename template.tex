\documentclass[12pt,a4paper]{article}
%Include graphics
\usepackage{graphicx}
\usepackage{graphics}
\usepackage{color}
\usepackage[dvipsnames, svgnames, x11names]{xcolor}
\usepackage{alltt}
\usepackage{url}
%Mathematic packages
\usepackage{mathptmx}
\usepackage{amsmath,amssymb,amsthm}
\usepackage{newtxmath}
\usepackage{textcomp,gensymb}% °
%tikz
\usepackage{tikz}
%Chinese
\usepackage{xeCJK}
\xeCJKsetup{EmboldenFactor=2.3}%假粗体加粗倍率
\newCJKfontfamily[fsong]\fangsong{FangSong}[AutoFakeBold,AutoFakeSlant]%仿宋体
\newCJKfontfamily[kai]\kaishu{KaiTi}[AutoFakeBold,AutoFakeSlant]%楷体
\newCJKfontfamily[hei]\heiti{SimHei}[AutoFakeBold,AutoFakeSlant]%黑体
\newCJKfontfamily[lisu]\lisu{LiSu}[AutoFakeBold,AutoFakeSlant]%隶书
\newCJKfontfamily[yyuan]\yyuan{YouYuan}[AutoFakeBold,AutoFakeSlant]%幼圆
% \setCJKmainfont{FZNewShuSong-Z10}[AutoFakeBold,AutoFakeSlant]
\setCJKmainfont{SimSun}[AutoFakeBold,AutoFakeSlant]
\usepackage{indentfirst}
\setlength{\parindent}{2em}
% \usepackage{zhspacing}
% \zhspacing
\linespread{1.083333333333333}%word单倍行距
\usepackage{zhnumber} % change section number to chinese
% \renewcommand\thesection{\zhnum{section}}
% \renewcommand\thesubsection{\arabic{subsection}}
\usepackage{titlesec}
% Changes the style of section headings
\titleformat{\section}[hang]{\heiti\Large\bfseries}{\zhnum{section}、}{0em}{}[]
\titleformat{\subsection}[hang]{\heiti\large\bfseries}{\arabic{subsection}}{0.5em}{}[]
\titleformat{\subsubsection}[hang]{\heiti\bfseries}{\arabic{subsection}.\arabic{subsubsection}}{0.5em}{}[]
\renewcommand{\contentsname}{目录}
\renewcommand{\figurename}{图}
\renewcommand{\tablename}{表}
%Code
\usepackage{listings}%Source code
\lstset{
	backgroundcolor=\color[RGB]{245,245,244},%代码块背景色为浅灰色
	rulesepcolor= \color{gray}, %代码块边框颜色
	breaklines=true,  %代码过长则换行
	numbers=left, %行号在左侧显示
	numberstyle= \small,%行号字体
	keywordstyle= \color{blue},%关键字颜色
	commentstyle=\color{gray}, %注释颜色
	frame=shadowbox%用方框框住代码块
}
% Set page margins
\usepackage[top=2.54cm,bottom=2.54cm,left=3.17cm,right=3.17cm]{geometry}
\usepackage{subcaption}
% Package used for placeholder text
\usepackage{lipsum}
\usepackage{zhlipsum}
% series of tables
\usepackage{longtable}
\usepackage{threeparttable}
\usepackage{booktabs}
\usepackage{multirow}
\usepackage{makecell}
% Prevents LaTeX from filling out a page to the bottom
\raggedbottom
% All page numbers positioned at the bottom of the page
\usepackage{lastpage}
\usepackage{fancyhdr}
\fancyhf{} % clear all header and footers
\fancyfoot[C]{\kaishu 第~\thepage~页,共~\pageref{LastPage}~页}
\renewcommand{\headrulewidth}{0pt} % remove the header rule
\pagestyle{fancy}

% Adds table captions above the table per default
\usepackage{float}
\floatstyle{plaintop}
\restylefloat{table}
% Adds space between caption and table
\usepackage[tableposition=top,font=small]{caption}
% hyper
\usepackage{hyperref}
\hypersetup{
	colorlinks=true,
	linkcolor=black,
	citecolor=black,
	urlcolor=black
}
\usepackage[capitalise]{cleveref}
\crefname{theorem}{定理}{定理}
\crefname{lemma}{引理}{引理}
\crefname{definition}{定义}{定义}
\crefname{figure}{图}{图}
\crefname{table}{表}{表}
\crefname{algorithm}{算法}{算法}
\usepackage{tcolorbox}
\graphicspath{ {Figures/} }
%%%%%%%%%%%%%%%%%%%%%%%%%%%%%% Starts the document
\begin{document}
\renewcommand{\today}{\number\year 年 \number\month 月}
%%%%% Adds the title page
\begin{titlepage}
    \clearpage\thispagestyle{empty}
    \centering
    \vspace{1cm}

    % Titles
    {\fangsong 课程号:PR00000}
    \vspace{2.5cm}

    \rule{\linewidth}{2mm} \\[0.5cm]
    { \Huge \kaishu \textbf{课程名}\\[0.2em]
    实习报告}\\[0.5cm]
    \rule{\linewidth}{0.6mm} \\[3.4cm]

    \hspace{2cm}
    \begin{tabular}{l p{5cm}}
        \textbf{专业} & 测绘工程   \\[18pt]
        \textbf{班级} & 10121911   \\[18pt]
        \textbf{学号} & 1012191100 \\[18pt]
        \textbf{姓名} & 舒己
    \end{tabular}

    \vspace{1.5cm}
    \centering \includegraphics[height=4.5cm]{校徽与中英文(蓝色中轴式).png}\\ % 
    \vspace{1.5cm}
    \begin{center}
        \today
    \end{center}
\end{titlepage}
%%%%%%%%%%%%%%%%%%正文%%%%%%%%%%%%%%%%%%%%%%%%%%%
\tableofcontents\thispagestyle{empty}
\newpage
\setcounter{page}{1}
\section{概述}

\section{一级标题}

\subsection{二级标题}

\subsubsection{三级标题}

\begin{table}[H]
    \centering
    \caption{\label{tab}三线表}
    \begin{tabular}{cc}
        \toprule
        建筑名称 & 测站数 \\
        \midrule
        逸夫楼   & 14     \\
        探工楼   & 10     \\
        综合楼   & 13     \\
        食堂     & 9      \\
        教二楼   & 15     \\
        教五楼   & 10     \\
        \bottomrule
    \end{tabular}
\end{table}

\begin{figure}[H]
    \centering
    \includegraphics[width=0.7\textwidth]{校徽与中英文(蓝色中轴式).png}
    \caption{\label{fig:yif_std}单图}
\end{figure}

\begin{figure}[H]
    \centering
    \begin{subfigure}{0.3\textwidth}
        \centering
        \includegraphics[width=\linewidth]{校徽与中英文(蓝色中轴式).png}
    \end{subfigure}
    \begin{subfigure}{0.3\textwidth}
        \centering
        \includegraphics[width=\linewidth]{校徽与中英文(蓝色中轴式).png}
    \end{subfigure}
    \begin{subfigure}{0.3\textwidth}
        \centering
        \includegraphics[width=\linewidth]{校徽与中英文(蓝色中轴式).png}
    \end{subfigure}
    \begin{subfigure}{0.3\textwidth}
        \centering
        \includegraphics[width=\linewidth]{校徽与中英文(蓝色中轴式).png}
    \end{subfigure}
    \begin{subfigure}{0.3\textwidth}
        \centering
        \includegraphics[width=\linewidth]{校徽与中英文(蓝色中轴式).png}
    \end{subfigure}
    \caption{\label{fig:boxr}多图}
\end{figure}
\end{document}
